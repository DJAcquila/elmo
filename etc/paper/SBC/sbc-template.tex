\documentclass[14pt]{article}

\usepackage{sbc-template}

\usepackage{graphicx,url}

%\usepackage[brazil]{babel}
\usepackage[utf8x]{inputenc}
\usepackage{booktabs}
\usepackage{multirow}
\usepackage{graphicx}
\usepackage{array}
\usepackage{todonotes}

%\usepackage[section]{placeins}

\setlength{\marginparwidth}{0cm}
\newcommand{\notaerick}[1]{\todo{[Erick:]#1}}

\sloppy

\title{Portuguese Word Embeddings: Evaluating on Word Analogies and Natural Language Tasks}

\author{Nathan S. Hartmann\inst{1}, Erick Fonseca\inst{1}, Christopher D. Shulby\inst{1},\\ Marcos V. Treviso\inst{1}, Jéssica S. Rodrigues\inst{2}, Sandra M. Aluísio\inst{1}}

\address{University of São Paulo, Institute of Mathematics and Computer Sciences
\nextinstitute
  Federal University of São Carlos, Department of Computer Science
\email{\{nathansh,erickrf,sandra\}@icmc.usp.br}
\email{\{chrisshulby,marcosvtreviso,jsc\}@gmail.com}
}

\begin{document}

\maketitle

\begin{abstract}
Word embeddings have been found to provide meaningful representations for words in an efficient way; therefore, they have become common in Natural Language Processing systems. In this paper, we evaluated different word embedding models trained on a large Portuguese corpus, including both Brazilian and European variants. We trained 31 word embedding models using FastText, GloVe, Wang2Vec and Word2Vec. We evaluated them intrinsically on syntactic and semantic analogies and extrinsically on POS tagging and sentence semantic similarity tasks. The obtained results suggest that word analogies are not appropriate for word embedding evaluation; task-specific evaluations appear to be a better option.
%The results are aligned with those obtained by \cite{repeval:16}.
\end{abstract}

\section{Introduction}

% Natural Language Processing (NLP) applications usually receive texts as an input, for this reason, words can be considered as basic processing units. Therefore, it is important that they are represented in some way that carries relevant information.
% As long more sophisticated NLP and machine learning techniques were developed, more effective representations of words appeared. The work of \cite{bengio2003neural} was one of the pioneers in employing neural networks to automatically learn vector representations. This type of representation is known as word embeddings. Word embeddings are able to capture semantic, syntactic and morphological information from large non-annotated corpus \cite{collobertetal2011,mikolovetal2013, Ling:2015:naacl, lai2015recurrent}.

Natural Language Processing (NLP) applications usually take words as basic input units; therefore, it is important that they be represented in a meaningful way.
In recent years, \emph{word embeddings} have been found to efficiently provide such representations, and consequently, have become common in modern NLP systems. They are vectors of real valued numbers, which represent words in an $n$-dimensional space, learned from large non-annotated corpora and able to capture syntactic, semantic and morphological knowledge.


%O uso de word embeddings é uma abordagem recente que vem obtendo sucesso, as quais possuem uma representação baseada em vetores reais distribuídos em um espaço multidimensional induzidos através de aprendizado não-supervisionado (TURIAN; RATINOV; BENGIO, 2010). Cada dimensão da embedding de uma palavra representa uma feature latente, de modo a capturarem distribuidamente propriedades semânticas, sintáticas ou morfológicas dessa palavra (COLLOBERT et al., 2011). O número de dimensões dos vetores pode variar, e em geral, melhores representações podem ser alcançadas com um maior número de dimensões. Mas se esse número for muito grande, o processamento de indução pode ser demorado.

%In order to apply neural networks in NLP tasks, it is necessary to map words of a text to a numeric vector or word embeddings. This representation allows to capture latent lexical and semantic relations \cite {collobertetal2011, mikolovetal2013}.

Different algorithms have been developed to generate embeddings \cite[\emph{inter alia}]{bengio2003neural, collobertetal2011,mikolovetal2013, Ling:2015:naacl, lai2015recurrent}. They can be roughly divided into two families of methods \cite{baronietal2014}: the first is composed of methods that work with a co-occurrence word matrix, such as Latent Semantic Analysis (LSA) \cite{dumais1988using}, Hyperspace Analogue to Language (HAL) \cite{lund1996producing} and Global Vectors (GloVe) \cite{penningtonetal2014}. The second is composed of predictive methods, which try to predict neighboring words given one or more context words, such as Word2Vec \cite{mikolovetal2013}.

%Word embeddings are being applied to many syntactic and semantic tasks such speech recognition \cite{mikolov2009neural}, semantic similarity \cite{mikolovetal2013}, part-of-speech (POS) tagging, sentiment analysis \cite{li2015multi} and logical semantics \cite{bowman2014recursive}. 

% \cite{mikolovetal2013} developed a benchmark of word embeddings evaluation. This benchmark is composed of instances of analogies which each instance contains 4 words. For example, Berlin, Germany, Lisbon, and Portugal what means ``Berlins is to Germany like Lisbon is to Portugal''. \cite{rodriguesetal2016} translated this benchmark to Portuguese and made it available\footnote{https://github.com/nlx-group/lx-dsemvectors}. This benchmark contains syntactic and semantic analogies. Syntactic analogies consist of adjectives to adverbs, antonyms, comparatives, superlatives, present participle, nationality adjectives, conditional verbs, nouns plural and verbs plural. Semantic analogies consist of common countries and their capitals, all countries and their capitals, currency, cities and states and family relationships.

%\todo[inline]{Primeira vez que fala sobre word embeddings e já entra comentando sobre benchmark de avaliação.}

%\todo[inline]{(Erick) Melhor assim?}

Given this variety of word embedding models, methods for evaluating them becomes a topic of interest. \cite{mikolovetal2013} developed a benchmark for embedding evaluation based on a series of analogies. Each analogy is composed of two pairs of words that share some syntactic or semantic relationship, e.g., the names of two countries and their respective capitals, or two verbs in their present and past tense forms. In order to evaluate an embedding model, applying some vectorial algebra operation to the vectors of three of the words should yield the vector of the fourth one. A version of this dataset translated and adapted to Portuguese was created by \cite{rodriguesetal2016}. %The benchmark contains five types of semantic analogy: (i) common capitals and countries, (ii) all capitals and countries, (iii) currency and countries, (iv) cities and states, and (v) family relations. Moreover, nine types of syntactic analogy are also represented: adjectives and adverbs, opposite adjectives, base adjectives and comparatives, base adjectives and superlatives, verb infinitives and present participles, countries and nationalities (adjectives), verb infinitives and past tense forms, nouns in plural and singular, and verbs in plural and singular. 
%The test set contains a total of 8869 semantic and 10675 syntactic entries.

However, in spite of being popular and computationally cheap, \cite{repeval:16} suggests that word analogies are not appropriate for evaluating embeddings. Instead, they suggest using task-specific evaluations, i.e., to compare word embedding models on how well they perform on downstream NLP tasks.

In this paper, we evaluated different word embedding models trained on a large Portuguese corpus, including both Brazilian and European variants (Section 2). %We used a different tokenization process than \cite{rodriguesetal2016}, and 
We trained our models using four different algorithms with varying dimensions (Section 3). We evaluated them on the aforementioned analogies as well as on POS tagging and sentence similarity, to assess both syntactic and semantic properties of the word embeddings (Section 4). Section 5 revises recent studies evaluating Portuguese word embeddings and compares literature results with ours. The contributions of this paper are: i) to make a set of 31 word embedding models publicly available\footnote{Available at \url{http://nilc.icmc.usp.br/embeddings}} as well as the script used for corpus preprocessing; and ii) an intrinsic and extrinsic evaluation of word embedding models, indicating the lack of correlation between performance in syntactic and semantic analogies and syntactic and semantic NLP tasks.

%In this paper, we focus is the different models of word embeddings in a large corpus of the Portuguese language in both Brazilian and Portuguese variants and their evaluation on syntactic and semantic analogies and syntactic and semantic tasks -- pos tagging and sentence similarity. Our contributions are: 1) make available a set of 32-word embeddings' models; 2) an intrinsic and an extrinsic evaluation of our models; 3) indications of non-correlation between the performance of a word embedding' model in syntactic and semantic analogies and syntactic and semantic tasks.

\section{Training Corpus}

% In this section, we present the corpus compiled for this paper and its preprocessing steps.
We collected a large corpus from several sources in order to obtain a multi-genre corpus, representative of the Portuguese language. We rely on the results found by \cite{rodriguesetal2016} and \cite{Fonseca2016} which indicate that the bigger a corpus is, the better the embeddings obtained, even if it is mixed with Brazilian and European texts. Table \ref{tab:corpusembeddings} presents all corpora collected in this work.

\subsection{Preprocessing}

We tokenized and normalized our corpus in order to reduce the vocabulary size, under the premise that vocabulary reduction provides more representative vectors. Word types with less than five occurrences were replaced by a special \texttt{UNKNOWN} symbol. %All HTML tags and elements between brackets were removed. (meio óbvio, não?)
Numerals were normalized to zeros;  URL's were mapped to a token \texttt{URL} and emails were mapped to a token \texttt{EMAIL}. 

Then, we tokenized the text relying on whitespaces and punctuation signs, paying special attention to hyphenation. Clitic pronouns like ``machucou-se'' are kept intact. Since it differs from the approach used in \cite{rodriguesetal2016} and their corpus is a subset of ours, we adapted their tokenization using our criteria. We also removed their Wikipedia section, and in all our subcorpora, we only used sentences with 5 or more tokens in order to reduce noisy content. This reduced the number of tokens of LX-Corpus from 1,723,693,241 to 714,286,638. 

%\todo[inline]{(Erick) o que isso quer dizer? "A huge vocabulary is undesirable because we need to represent an amount of words in a fixed vector size"}

%\todo[inline]{RE-(Nathan) temos um espaço limitado para representar uma quantidade variante de palavras. Quando maior for essa quantidade, menos o poder de representação dos vetores.}

%\todo[inline]{(erick) tirei a fatídica frase, estava muito confusa e acho que não faz falta.}

%\cite{rodriguesetal2016} showed that the larger the corpus, the better. We also rely on results of \cite{Fonseca2016} to collect corpora of both Brazilian and European variants of Portuguese in order to have available as many texts as possible. Table \ref{tab:corpusembeddings} presents all corpora collected in this work. \textbf{All genre presented follow L\'acioWeb pattern\footnote{http://143.107.183.175:22180/lacioweb/tipologia.htm} \cite{aluisio2003lacioweb}}


% (erick) comentei a lista pq já temos a tabela que é mais organizada
%\begin{itemize}
    %\item LX-Corpus \cite{rodriguesetal2016}:
    %\item Wikipedia: It is a Wikipedia's dump of 10/20/16. A preview's version was used by \cite{hartmann2016} whose obtained best result in ASSIN sentence similarity shared-task.
    %\item GoogleNews:
    %\item SubIMDB-PT: It is the result of a crawling of subtitles from IMDB website in 12/03/2016\footnote{http://www.imdb.com}.
    %\item G1: It is the result of a crawling on g1 website\footnote{http://g1.globo.com}. It was already used by \cite{hartmann2016}.
    %\item PLN-Br \cite{bruckschenetal2008}: It is a corpus largely used on Brazilian Portuguese NLP research. It was used by \cite{hartmann2016} to train word embeddings' models.
    %\item Literacy works of public domain: A collection of 138,268 literacy works from Domínio Público website\footnote{http://www.dominiopublico.gov.br}.
    %\item Lacioweb: It is a corpus compiled in the Lácio-Web project \cite{aluisio2003lacioweb} composed by well written Brazilian Portuguese texts. This corpus has informative, scientific, poetry, prose, and drama texts.
    %\item Portuguese books
    %\item Mundo Estranho: It is the result of a crawling on Mundo Estranho website\footnote{http://mundoestranho.abril.com.br}.
    %\item It is the result of a crawling of Ci\^encia Hoje das Crian\c{c}as (CHC) website\footnote{chc.cienciahoje.uol.com.br}.
    %\item FAPESP: It is the corpus Revista Pesquisa FAPESP.
    %\item Textbooks: A collection of texts extracted from textbooks written for children between 3rd and 7th-grade years of primary school.
    %\item Folhinha: \footnote{www.folha.uol.com.br/folhinha}
    %\item NILC corpus: \footnote{nilc.icmc.usp.br/nilc/images/download/corpusNilc.zip}
    %\item Para Seu Filho Ler: \footnote{zh.clicrbs.com.br/rs}
    %\item SARESP: \footnote{sites.google.com/site/provassaresp}
%\end{itemize}

% \todo[inline]{(erick) Sobre a tabela:

%1) Acabei de verificar que a publicação original do LX-vectors usou um corpus de 1.7 bilhões de tokens. Mesmo com diferenças de tokenização, ainda é bem maior que o nosso! Precisa incluir no texto que pegamos menos da metade do corpus deles e por quê.


\begin{table}[!ht]
  \centering
    \scriptsize
    \scalebox{.9}{
    \begin{tabular}{m{2.7cm}llm{2cm}m{7cm}}
      \toprule
        \textbf{Corpus} & \textbf{Tokens} & \textbf{Types} & \textbf{Genre} & \textbf{Description}\\
        \midrule
        LX-Corpus \cite{rodriguesetal2016} & 714,286,638 & 2,605,393 & Mixed genres & A huge collection of texts from 19 sources. Most of them are written in European Portuguese. \\%This corpus was used in an extensive evaluation of corpus combination and hyperparameters tuning \cite{rodriguesetal2016}\\
        \midrule
        Wikipedia & 219,293,003 & 1,758,191 &  Encyclopedic & Wikipedia dump of 10/20/16 \\
        %A preview's version was used by \cite{hartmann2016} whose obtained best result in ASSIN sentence similarity shared-task\footnote{http://propor2016.di.fc.ul.pt/?page\_id=381}\\
        \midrule
        GoogleNews & 160,396,456 & 664,320 & Informative &  News crawled from GoogleNews service\\%\footnote{https://news.google.com.br} between 2014 and 2016\\
        \midrule
        SubIMDB-PT & 129,975,149 & 500,302 & Spoken language & Subtitles crawled from IMDb website\\%\footnote{http://www.imdb.com} in 12/03/2016\\
        \midrule
        G1 & 105,341,070 & 392,635 & Informative & News crawled from G1 news portal between 2014 and 2015.\\%\footnote{http://g1.globo.com}% It was already used by \cite{hartmann2016}\\
        \midrule
        PLN-Br \cite{bruckschenetal2008} & 31,196,395 & 259,762 & Informative & Large corpus of the PLN-BR Project with texts sampled from  1994 to 2005. It was also used by \cite{hartmann2016} to train word embeddings models\\
        \midrule
        Literacy works of\newline public domain & 23,750,521 & 381,697 & Prose & A collection of 138,268 literary works from the Domínio Público website \\%\footnote{http://www.dominiopublico.gov.br}\\
        \midrule
        Lacio-web \cite{aluisio2003lacioweb} & 8,962,718 & 196,077 & Mixed genres & Texts from various genres, e.g., literary and its subdivisions (prose, poetry and drama), informative, scientific, law, didactic technical\\
        %, textual types (e.g. article, manual, research project, letter, biography), subjects (e.g. politics, environment, life style, sports, arts, religion)\\
%Corpus composed of informative, scientific, poetry, prose, and drama texts  
        \midrule
        Portuguese e-books & 1,299,008 & 66,706 & Prose & Collection of classical fiction books written in Brazilian Portuguese crawled from Literatura Brasileira website\\% (p.e., \textit{Iracema}, \textit{O Ateneu} and \textit{O Cortiço})\\
        \midrule
        Mundo Estranho & 1,047,108 & 55,000 & Informative & Texts crawled from Mundo Estranho magazine\\%\footnote{http://mundoestranho.abril.com.br}\\
        \midrule
        CHC & 941,032 & 36,522 & Informative & Texts crawled from Ci\^encia Hoje das Crian\c{c}as (CHC) website\\%\footnote{chc.cienciahoje.uol.com.br}\\
        \midrule
        FAPESP & 499,008 & 31,746 & Science \newline Communication & Brazilian science divulgation texts from Pesquisa FAPESP magazine\\
        \midrule
        Textbooks & 96,209 & 11,597 & Didactic & Texts for children between 3rd and 7th-grade years of elementary school\\
        \midrule
        Folhinha & 73,575 & 9,207 & Informative & News written for children, crawled in 2015 from Folhinha issue of Folha de São Paulo newspaper\\%\footnote{www.folha.uol.com.br/folhinha}.\\
        \midrule
        NILC subcorpus & 32,868 & 4,064 & Informative & Texts written for children of 3rd and 4th-years of elementary school  \\%\footnote{nilc.icmc.usp.br/nilc/images/download/corpusNilc.zip}
        \midrule
        Para Seu Filho Ler & 21,224 & 3,942 & Informative & News written for children, from Zero Hora newspaper \\%\footnote{zh.clicrbs.com.br/rs}\\
        \midrule
        SARESP & 13,308 & 3,293 &  Didactic & Text questions of Mathematics, Human Sciences, Nature Sciences and essay writing to evaluate students\\
        
        %of the 3rd, 5th, 7th and 9th years of elementary school and the 3rd grade of the High School of the State of São Paulo \\%\footnote{sites.google.com/site/provassaresp}\\
        \bottomrule
        \\
        \cmidrule{1-3}
        \textbf{Total} & 1,395,926,282 & 3,827,725\\
        \cmidrule{1-3}
    \end{tabular}}
    \caption{Sources and statistics of corpora collected.}
    \label{tab:corpusembeddings}
\end{table}

%\todo[inline]{(Erick) Não dá para resumir um pouco as descrições?? Estamos gastando espaço demais com isso, inclusive para corpora muito pequenos}



\section{Embedding Methods}

In this section, we describe the four methods we used to train 31 word embedding models: GloVe, Word2Vec, Wang2Vec, and FastText.

%\todo[inline]{Favor, descrever o diferencial de cada modelo. Qual sua característica que o torna ``especial''?}
%\todo[inline]{Deixei em negrito o diferencial!}
%We trained word embeddings' instances of 50, 100, 300, 600 and 1,000 dimensions for Word2vec methods.


%\subsection{GloVe}

The Global Vectors (GloVe) method was proposed by \cite{penningtonetal2014}, and obtained state-of-the-art results for \emph{syntactic} and \emph{semantic} analogies tasks. This method consists in a co-occurrence matrix $M$ that is constructed by looking at context words. Each element $M_{ij}$ in the matrix represents the probability of the word $i$ being close to the word $j$. In the matrix $M$, the rows (or vectors) are randomly generated and trained by obeying the equation $P(w_i, w_j) = log(M_{ij}) = w_iw_j + b_i + b_j$
%Equation \ref{Eq:GloVe:Constraiment}
, where $w_i$ and $w_j$ are word vectors, and $b_i$ and $b_j$ are biases.



% \begin{equation}
% \label{Eq:GloVe:Constraiment}
% \textbf{M}_{ij} = \textbf{w}_i\textbf{w}_j + b_i + b_j
% \end{equation}


%\subsection{Word2Vec}

%Word2Vec is a widely used method in NLP that explores the idea of using a neural network\notaerick{não é uma rede neural de verdade. não tem camada oculta} to induce dense representation of a word \cite{collobertetal2011}, where the network does not have a hidden layer, resulting in a fast log-linear model \cite{mikolovetal2013}. This network can be divided in two architectures: (i) \textit{Continuous Bag-of-Words (CBOW)}, where given a sequence of words the model attempts to predict the word in the middle; (ii) \textit{Skip-Gram}, where given a word the model attempts to predict its neighboring words.

%\todo[inline]{Nova versão (Erick) segue abaixo}

Word2Vec is a widely used method in NLP for generating word embeddings. It has two different training strategies: (i) \emph{Continuous Bag-of-Words (CBOW)}, in which the model is given a sequence of words without the middle one, and attempts to predict this omitted word; (ii) \emph{Skip-Gram}, in which the model is given a word and attempts to predict its neighboring words. In both cases, the model consists of only a single weight matrix (apart from the word embeddings), which results in a fast log-linear training that is able to capture \emph{semantic} information~\cite{mikolovetal2013}. 

% It was already shown by the original works \cite{penningtonetal2014,mikolovetal2013} that GloVe and Word2Vec are able to capture \textbf{semantic} information in their induced embeddings.

% Seria bom falar que o Wang2Vec foi proposto para capturar similaridade sintaticas
%\subsection{Wang2Vec}

% Since Word2Vec does not consider word order, Wang2Vec  \cite{Ling:2015:naacl} was proposed as a modification in the CBOW architecture. It concatenates the input word vectors.

% \todo[inline]{reescrever como isso funciona}
% vou escrever em ptbr, dps traduzo

Wang2Vec is a modification of Word2Vec made in order to take into account the lack of word order in the original architecture. Two simple modifications were proposed in Wang2Vec expecting embeddings to better capture \emph{syntactic} behavior of words \cite{Ling:2015:naacl}. In the \emph{Continuous Window} architecture, the input is the concatenation of the context word embeddings in the order they occur. In \emph{Structured Skip-Gram}, a different set of parameters is used to predict each context word, depending on its position relative to the target word. %Then, the network has positional information of a word and its neighbors to induce word embeddings.

% \cite{Ling:2015:naacl}


% Foi proposto para capturar similaridades morfologicas. Além disso, é bom para palavras OOV, já que essas palavras possuem ngramas de chars
%\subsection{FastText}%

FastText is a recently developed method \cite{bojanowski2016enriching,joulin2016bag} in which embeddings are associated to character n-grams, and words are represented as the summation of these representations. In this method, a word representation is induced by summing character n-gram vectors with vectors of surrounding words. Therefore, this method attempts to capture \emph{morphological} information to induce word embeddings.

%\todo[inline] {Está meio confusa a explicação de como o vetor de uma palavra é criado.}
%\todo[inline] {É difícil explicar sem usar notação matemática}
%\todo[inline]{Acho melhor usar equações mesmo}

% \cite{bojanowski2016enriching}
\section{Evaluation}

In order to evaluate the robustness of the word embedding models we trained, we performed intrinsic and extrinsic evaluations. For the intrinsic evaluation, we used the set of syntactic and semantic analogies from \cite{rodriguesetal2016}. For extrinsic evaluation, we chose to apply the trained models on POS tagging and sentence similarity tasks. The tasks were chosen deliberately since they are linguistically aligned with the sets of analogies used in the first evaluation. POS tagging is by nature a morphosyntactic task, and although some analogies are traditionally regarded as \emph{syntactic}, they are actually morphological --- for example, suffix operations. Sentence similarity is a semantic task since it evaluates if two sentences have similar meaning. It is expected that the models that achieve the best results in syntactic (morphological) analogies also do so in POS tagging, and the same is true for semantic analogies and semantic similarity evaluation. We trained embeddings with the following dimensions numbers: 50, 100, 300, 600 and 1,000.
 %\todo[inline]{(erick) não entendi pq no parágrafo acima está dito que consideramos POS tagging uma tarefa sintática. Seria para fazer um paralelo com as avaliações intrínsecas sinático/semânticas??? Se for, não tem nada a ver. A avaliação intrínseca chamada carinhosamente de sintática é puramente morfológica (são palavras isoladas!!!); e de qualquer forma, não precisamos nos limitar a fazer paralelos da avaliação intrínseca com a extrínseca.}

%We trained embeddings with the following dimensions numbers: 50, 100, 300, 600 and 1,000. It should be noted that we suffered from RAM restrictions and, for some models, it was not possible to train word embeddings of higher dimensions.


\subsection{Intrinsic evaluation}

We evaluated our embeddings in the syntactic and semantic analogies provided by \cite{rodriguesetal2016}. Since our corpus is composed of both Brazilian (PT-BR) and European (PT-EU) Portuguese, we also evaluated the models in the test sets for both variants, following \cite{rodriguesetal2016}.

Table \ref{tab:evaluation} shows the obtained results for the intrinsic evaluation. On average, GloVe was the best model for both Portuguese variants. The model which best performed on syntactic analogies was FastText, followed by Wang2Vec. This makes sense since FastText is a morphological model, and Wang2Vec uses word order, which provides some minimal syntactic knowledge. In semantic analogies, the model which best performed was GloVe, followed by Wang2Vec. GloVe is known for modeling semantic information well. Wang2Vec potentially captures semantics because it uses word order. The position of a negation in a sentence can totally change its semantics. If this negation is shuffled in a bag of words (Word2Vec CBOW), sentence semantic is diluted.
%\todo[inline]{(nathan) inserir um exemplo de negação em uma sentença para mostrar como a posição dela influencia diretamente na semântica da sentença. No caso do bag-of-words, não se sabe a posição da negação e perdemos poder semântico.}
%\todo[inline]{(erick) o fato de o wang2vec usar word order foi dado como explicação para sua performance nos dois modelos. para o "sintático" até vai, mas para o semântico não faz sentido sem elaborar mais. Chris to Carioca: talvez pega semantica pelo conexto, pode elaborar mais nese sentido pois creio que vai disambiguar a semantica pelas palavras vizinhas, para o argumento sintatico as dependencias podem ser inferidas nas mesmas janelas. Nathan: a ordem é extremamente importante para a semântica. O local onde uma negação ocorre pode alterar completamente a semântica do contexto.}

All CBOW models, except for the Wang2Vec ones, achieved very low results in semantic analogies, similarly to the results from \cite{mikolovetal2013}. %It is interesting because CBOW uses information from the surrounding words in its prediction, and contextual information should be useful in semantics.
%\notaerick{discordo, isso é mais útil para sintaxe. opinao do Chris: da para disambiguar a semantica com contexto como no trabalho do fernando, claro que a sintaxe vai aproveitar ainda mais mas a semantica vai se benificiar bastante. Nathan: a ordem é extremamente importante para a semântica. O local onde uma negação ocorre pode alterar completamente a semântica do contexto.}
%However, Wang2Vec CBOW achieved semantic results comparable to the good ones achieved by other methods. 
Wang2Vec CBOW differs from other CBOW methods in that it takes word order into account, and then we can speculate that an unordered bag-of-words is not able to capture a word's semantic so well.

%\todo[inline]{(erick) Faltou mencionar como isso se compara com a literatura! no trabalho original do mikolov, já havia exatamente isso: o CBOW teve resultados muito ruins em analogias semânticas. Tendo isso em mente, o interessante desses resultados é o wang2vec cbow ter obtido bons resultados. (SANDRA) Erick, escreva !!! O artigo é seu também. Você tem um ótimo inglês :)}
%We also compared our trained models with the best model of \cite{sousa2016}, a 300 dimensions CBOW, in despite of the author just evaluated for Brazilian Portuguese. Table \ref{tab:evaluation} shows obtained results.


\begin{table}[t]
	\center
    \footnotesize
    \scalebox{.73}{
    \begin{tabular}{llrccc|ccc}
    	\toprule
        \multicolumn{2}{c}{\multirow{2}{*}{\textbf{Embedding Models}}} & \multirow{2}{*}{\textbf{Size}} & \multicolumn{3}{c}{\textbf{PT-BR}} & \multicolumn{3}{c}{\textbf{PT-EU}}\\
        \cmidrule{4-9}
        \multicolumn{2}{c}{} & & \textbf{Syntactic} & \textbf{Semantic} & \textbf{All} & \textbf{Syntactic} & \textbf{Semantic} & \textbf{All}\\
        \midrule
        & & 50 	                & 35.2 & 4.2 & 19.6 & 35.2 & 4.6 & 19.8 \\
        %\cmidrule{3-9}
        & & 100                 & 45.0 & 6.1 & 25.5 & 45.1 & 6.4 & 25.7  \\
        %\cmidrule{3-9}
        & CBOW & 300            & 52.0 & 8.4 & 30.1 & 52.0 & 9.1 & 30.5  \\
        %\cmidrule{3-9}
        \multirow{2}{*}{FastText} & & 600                 & 52.6 & 5.9 & 29.2 & 52.4 & 6.5 & 29.4\\
        & & 1,000 & 50.6 & 4.8 & 27.7 & 50.4 & 5.4 & 27.9\\
        \cmidrule{2-9}
        &  & 50                 & 36.8 & 18.4 & 27.6 & 36.5 & 17.1 & 26.8  \\
        %\cmidrule{3-9}
        &  & 100 & 50.8 & 30.0 & 40.4 & 50.7 & 28.9 & 39.8   \\
        %\cmidrule{3-9}
        & Skip-Gram & 300 & \textbf{58.7} & 32.2 & 45.4 & \textbf{58.5} & 31.1 & 44.8  \\
        %\cmidrule{3-9}
        & & 600 & 55.1 & 24.3 & 39.6 & 55.0 & 23.9 & 39.4\\
        & & 1,000 & 45.1 & 14.6 & 29.8 & 45.2 & 13.8 & 29.4\\
        \midrule         
        &  & 50 & 28.7 & 13.7 & 27.4 & 28.5 & 12.8 & 27.7\\
        %\cmidrule{3-9}
         & & 100 & 39.7 & 28.7 & 34.2 & 39.9 & 26.6 & 33.2\\
        %\cmidrule{3-9}
        GloVe & & 300 & 45.8 & 45.8 & \textbf{46.7} & 45.9 & 42.3 & \textbf{46.2}\\
        %\cmidrule{3-9}
        %& & 400 & 43.8 & \textbf{46.9} & 45.4 & 44.2 & \textbf{43.0} & 43.6\\
        & & 600 & 42.3 & \textbf{48.5} & 45.4 & 42.3 & \textbf{43.8} & 43.1\\
        & & 1,000 & 39.4 & 45.9 & 42.7 & 39.8 & 42.5 & 41.1\\
        \midrule
		 & & 50 & 28.4 & 9.2 & 18.8 & 28.4 & 8.9 & 18.6\\
        %\cmidrule{3-9}
        &  & 100 & 40.9 & 26.2 & 33.5 & 40.8 & 24.4 & 32.6\\
        %\cmidrule{3-9}
         & CBOW & 300 & 49.9 & 40.3 & 45.1 & 50.0 & 36.9 & 43.5\\
        & & 600 & 46.1 & 22.2 & 34.1 & 46.0 & 21.1 & 33.5\\
        \multirow{2}{*}{Wang2Vec} & & 1,000 & 44.8 & 21.9 & 33.3 & 44.7 & 20.5 & 32.6\\
        %\cmidrule{3-9}
        %\multirow{2}{*}{Wang2Vec} & & 400 & & & & & &\\
        \cmidrule{2-9}
        &  & 50 & 30.6 & 12.2 & 21.3 & 30.6 & 11.5 & 21.0\\
        %\cmidrule{3-9}
        &  & 100 & 43.9 & 22.2 & 33.0 & 44.0 & 21.2 & 32.6 \\
        %\cmidrule{3-9}
        & Skip-Gram& 300 & 53.3 & 33.9 &  42.8 & 53.4 & 32.3 & 43.6\\
        %\cmidrule{3-9}
        %& & 400 & & & & & &\\
        & & 600 & 52.9 & 35.0 & 43.9 & 53.0 & 33.2 & 43.1\\
        & & 1,000 & 47.3 & 33.2 & 40.2 & 47.6 & 30.9 & 39.2\\
        \midrule
    	% & & 300 \cite{sousa2016} & 21.7 & 17.2 & 20.4 & - & - & -\\
        %\cmidrule{3-9}
        & & 50 & 9.8 & 2.2 & 6.0 & 9.7 & 1.9 & 5.8\\
        %\cmidrule{3-9}
        & & 100 & 16.2 & 3.6 & 9.9 & 16.0 & 3.5 & 9.7\\
        %\cmidrule{3-9}
        & CBOW & 300 & 24.7 & 4.6 & 23.9 & 24.5 & 4.5 & 23.6\\
        %\cmidrule{3-9}
        & & 600 & 25.8 & 5.2 & 23.1 & 25.4 & 5.1 & 22.9\\
        %\cmidrule{3-9}
        \multirow{2}{*}{Word2Vec} & & 1,000 & 26.2 & 4.9 & 22.9 & 26.2 & 4.5 & 22.7\\
        \cmidrule{2-9}
        &  & 50 & 17.0 & 5.4 & 11.2 & 16.9 & 4.8 & 10.8\\
        %\cmidrule{3-9}
        & & 100 & 25.2 & 8.0 & 16.6 & 24.8 & 7.4 & 16.1 \\
        %\cmidrule{3-9}
        & Skip-Gram & 300 & 33.0 & 15.6 & 29.2 & 32.2 & 14.1 & 29.8\\
        %\cmidrule{3-9}
       % & & 600 \cite{hartmann2016} & 33.0 & 17.9 & 25.5 & 32.9 & 15.5 & 24.2\\
        %\cmidrule{3-9}
        & & 600 & 35.6 & 20.0 & 33.4 & 35.3 & 17.6 & 33.5\\
        %\cmidrule{3-9}
        & & 1,000 & 34.1 & 21.3 & 32.6 & 33.6 & 18.1 & 31.9\\
        \bottomrule
    \end{tabular}}
    \caption{Intrinsic evaluation on syntactic and semantic analogies.}
    \label{tab:evaluation}
\end{table}


\subsection{Extrinsic Evaluation}

In this section we describe the experiments performed on POS tagging and Semantic Similarity tasks.

\subsubsection*{POS Tagging}

  POS tagging is a very suitable NLP task to evaluate how well the embeddings capture morphosyntactic properties. The two key difficulties here are: i) correctly classifying words that can have different tags depending on context; and ii) generalizing to previously unseen words.
    Our experiments were performed with the nlpnet POS tagger\footnote{More info at \url{http://nilc.icmc.usp.br/nlpnet/}} using the revised Mac-Morpho corpus and similar tagger configurations to those presented by \cite{Fonseca2015} (20 epochs, 100 hidden neurons, learning rate starting at 0.01, capitalization, suffix and prefix features). We did not focus on optimizing hyperparameters; instead, we set a single configuration to compare embeddings.
    
    Table~\ref{tab:pos} presents the POS accuracy results\footnote{Note that accuracies are well below those reported by \cite{Fonseca2015}. The probable cause is that the embedding vocabularies used here did not have clitic pronouns split from verbs, resulting in a great amount of out of vocabulary words.}.  As a rule of thumb, the larger the dimensionality, the better the performance. The exception is the 1,000 dimensions Word2Vec models, which performed slightly worse than those with 600. GloVe and FastText yielded the worst results, and Wang2Vec achieved the best. GloVe's poor performance may be explained by its focus on semantics rather than syntax, and FastText's performance was surprising in that despite its preference for morphology, something traditionally regarded as important for POS tagging, it yielded relatively poor results. Wang2Vec resulted in the best performance -- actually, its 300 dimension Skip-Gram model was superior to Word2Vec's 1000 model. Concerning the CBOW and Skip-Gram strategies, in the case of FastText, the latter was considerably better. For Wang2Vec and Word2Vec, the gap between the two is less noticeable, where CBOW achieved slightly better performance on smaller dimensionalities. 

\begin{table}[htb]
\centering
\footnotesize
\scalebox{.73}{
\begin{tabular}[t]{@{}llrr@{}}
\toprule
\multicolumn{2}{c}{\textbf{Embedding Models}}          & \multicolumn{1}{c}{\textbf{Size}} & \multicolumn{1}{c}{\textbf{Accuracy}} \\ \midrule
\multirow{10}{*}{FastText} & \multirow{5}{*}{CBOW}      & 50                                & 91.18\%                               \\
                          &                            & 100                               & 92.57\%                               \\
                          &                            & 300                               & 93.86\%                               \\
                          &                            & 600                               & 93.86\%                               \\ 
                          & & 1000 & 94.27\% \\
                          \cmidrule(l){2-4} 
                          & \multirow{5}{*}{Skip-Gram} & 50                                & 93.15\%                               \\
                          &                            & 100                               & 93.78\%                               \\
                          &                            & 300                               & 94.82\%                               \\
                          &                            & 600                               & 95.25\%                               \\
                          & & 1000 & 95.49\% \\ \midrule
 & \multirow{3}{*}{CBOW}      & 50                                & 95.33\%                               \\
                          &                            & 100                               & 95.59\%                               \\
\multirow{2}{*}{Wang2Vec}                          &                            & 300                               & 95.83\%                               \\ \cmidrule(l){2-4} 
                          & \multirow{5}{*}{Skip-Gram} & 50                                & 95.07\%                               \\
                          &                            & 100                               & 95.57\%                               \\
                          &                            & 300                               & 95.89\%                               \\
 &  & 600                               &       95.88\%                         \\
 & & 1,000 & \textbf{95.94\%} \\
\bottomrule
\end{tabular}}
\quad
\scalebox{0.73}{
\begin{tabular}[t]{@{}llrr@{}}
\toprule
\multicolumn{2}{c}{\textbf{Embeddings model}}           & \multicolumn{1}{c}{\textbf{Size}} & \multicolumn{1}{c}{\textbf{Accuracy}} \\ \midrule
\multirow{5}{*}{GloVe}     & \multirow{4}{*}{}         & 50                                & 93.13\%                               \\
                           &                            & 100                               & 93.72\%                               \\
                           &                            & 300                               & 94.76\%                               \\
                           &                            %& 400                               & 94.95\%                               \\ 
			& 600                               & 95.23\%\\
            & & 1,000                               & 95.57\%\\
                           \midrule
\multirow{10}{*}{Word2Vec} & \multirow{5}{*}{CBOW}      & 50                                & 95.00\%                               \\
                           &                            & 100                               & 95.27\%                               \\
                           &                            & 300                               & 95.58\%                               \\
                           &                            & 600                               & 95.65\%                               \\
                           &                            & 1,000                              & 95.62\%                               \\ \cmidrule(l){2-4} 
                           & \multirow{5}{*}{Skip-Gram} & 50                                & 94.79\%                               \\
                           &                            & 100                               & 95.18\%                               \\
                           &                            & 300                               & 95.66\%                               \\
                           &                            & 600                               & 95.82\%                               \\
                           &                            & 1,000                              & 95.81\%                               \\ \bottomrule
\end{tabular}}
\caption{Extrinsic evaluation on POS tagging}
\label{tab:pos}
\end{table}


\subsubsection*{Semantic Similarity}

ASSIN (\textit{Avaliação de Similaridade Semântica e Inferência Textual}) was a workshop co-located with PROPOR-2016. ASSIN made two shared-tasks available: i) semantic similarity; and ii) entailment. We chose the first one to evaluate our word embedding models extrinsically in a semantic task. ASSIN semantic similarity shared task required participants to assign similarity values between 1 and 5 to pairs of sentences. The workshop made training and test sets for Brazilian (PT-BR) and European (PT-EU) Portuguese available. \cite{hartmann2016} obtained the best results for this task. The author calculated the semantic similarity of pairs of sentences training a linear regressor with two features: i) the cosine similarity between the TF-IDF of each sentence; and ii) the cosine similarity between the summation of the word embeddings of the sentences' words. We chose this work as a baseline for evaluation because we can replace its word embedding model with others and compare the results. Although the combination of TF-IDF and word embeddings produced better results than only using word embeddings, we chose to only use embeddings for ease of comparison. \cite{hartmann2016} trained the word embedding model using Word2Vec Skip-Gram approach, with 600 dimensions, and a corpus composed of Wikipedia, G1 and PLN-Br. Only using embeddings, \cite{hartmann2016} achieved 0.58 in Pearson's Correlation ($\rho$) and a 0.50 Mean Squared Error (MSE) for PT-BR; and 0.55 $\rho$ and 0.83 MSE for PT-EU evaluation.

Table \ref{tab:semantic_similarity_evaluation} shows the performance of our word embedding models for both PT-BR and PT-EU test sets. To our surprise, the word embedding models which achieved the best results on semantic analogies (see Table \ref{tab:evaluation}) were not the best in this semantic task. The best results for European Portuguese was achieved by Word2Vec CBOW model using 1,000 dimensions. CBOW models were the worst on semantic analogies and were not expected to achieve the best results in this task. The best result for Brazilian Portuguese was obtained by Wang2Vec Skip-Gram model using 1,000 dimensions. This model also achieved the best results for POS tagging. Neither FastText nor GloVe models beat the results achieved by \cite{hartmann2016}.

% \begin{table}[!ht]
% \center
% \footnotesize
% \scalebox{1}{
% \begin{tabular}{llrrr|rr}
% \toprule
% \multicolumn{2}{c}{\multirow{2}{*}{\textbf{Embeddings Model}}} & \multirow{2}{*}{\textbf{Size}} & \multicolumn{2}{c}{\textbf{PT-BR}} & \multicolumn{2}{c}{\textbf{PT-PT}} \\
% \cmidrule{4-7}
%  \multicolumn{2}{c}{} & & \textbf{CP} & \textbf{EQM} & \textbf{CP} & \textbf{EQM}\\
% \midrule
% &  & 50 & 0.36 & 0.66 & 0.34 & 1.05\\
% %\cmidrule{3-7}
% & \multirow{2}{*}{CBOW} & 100 & 0.37 & 0.66 & 0.36 & 1.04\\
% %\cmidrule{3-7}
% & & 300 & 0.38 & 0.65 & 0.37 & 1.03\\
% %\cmidrule{3-7}
% \multirow{2}{*}{FastText} & & 600 & 0.33 & 0.68 & 0.38 & 1.02\\
% \cmidrule{2-7}
% & & 50 & 0.45 & 0.61 & 0.43 & 0.98\\
% %\cmidrule{3-7}
% & \multirow{2}{*}{Skip-Gram} & 100 & 0.49 & 0.58 & 0.47 & 0.94\\
% %\cmidrule{3-7}
% & & 300 & 0.55 & 0.53 & 0.40 & 1.02\\
% %\cmidrule{3-7}
% & & 600 & 0.40 & 0.64 & 0.40 & 1.01\\
% \midrule
% & & 50 & 0.42 & 0.62 & 0.38 & 1.01\\
% %\cmidrule{3-7}
% \multirow{2}{*}{GloVe} & & 100 & 0.45 & 0.60 & 0.42 & 0.98\\
% %\cmidrule{3-7}
% & & 300 & 0.49 & 0.58 & 0.45 & 0.95\\
% %\cmidrule{3-7}
% & & 400 & 0.50 & 0.57 & 0.46 & 0.95\\
% \midrule
% &  & 50 & 0.53 & 0.55 & 0.51 & 0.89\\
% %\cmidrule{3-7}
% & CBOW & 100 & 0.56 & 0.52 & 0.54 & \textbf{0.85}\\
% %\cmidrule{3-7}
% \multirow{2}{*}{Wang2Vec} & & 300  & 0.53 & 0.55 & 0.51 & 0.89\\
% \cmidrule{2-7}
% & & 50 & 0.51 & 0.56 & 0.47 & 0.92\\
% %\cmidrule{3-7}
% & Skip-Gram & 100 & 0.54 & 0.54 & 0.50 & 0.89\\
% %\cmidrule{3-7}
% & & 300 & \textbf{0.58} & \textbf{0.50} & 0.53 & \textbf{0.85}\\
% \midrule
% &  & 50 & 0.47 & 0.59 & 0.46 & 0.95\\
% %\cmidrule{3-7}
% &  & 100 & 0.50 & 0.57 & 0.49 & 0.91\\
% %\cmidrule{3-7}
% & CBOW & 300 & 0.55 & 0.53 & 0.54 & 0.87\\
% %\cmidrule{3-7}
% & & 600 & 0.57 & 0.51 & \textbf{0.55} & 0.86\\
% %\cmidrule{3-7}
% \multirow{2}{*}{Word2Vec} & & 1000 & \textbf{0.58} & \textbf{0.50} & \textbf{0.55} & 0.86\\
% \cmidrule{2-7}
% & & 50 & 0.46 & 0.60 & 0.43 & 0.97\\
% %\cmidrule{3-7}
% & & 100 & 0.48 & 0.58 & 0.45 & 0.95\\
% %\cmidrule{3-7}
% & \multirow{2}{*}{Skip-Gram} & 300 & 0.52 & 0.56 & 0.48 & 0.93\\
% %\cmidrule{3-7}
% %& & 600 \cite{hartmann2016} & 0.58 & 0.50 & \textbf{0.55} & 0.83\\
% %\cmidrule{3-7}
% & & 600 & 0.53 & 0.54 & 0.50 & 0.92\\
% %\cmidrule{3-7}
% & & 1000 & 0.54 & 0.54 & 0.50 & 0.91\\
% \bottomrule
% \end{tabular}}
% \vspace{0.1cm}
% \caption{ Extrinsic evaluation of word embeddings models on Semantic Similarity task.}
% \label{tab:semantic_similarity_evaluation}
% \end{table}






\begin{table}[htb]
\centering
\footnotesize
\scalebox{0.73}{
\begin{tabular}[t]{llrrr|rr}
\toprule
\multicolumn{2}{c}{\multirow{2}{*}{\textbf{Embedding Models}}} & \multirow{2}{*}{\textbf{Size}} & \multicolumn{2}{c}{\textbf{PT-BR}} & \multicolumn{2}{c}{\textbf{PT-EU}} \\
\cmidrule{4-7}
 \multicolumn{2}{c}{} & & \textbf{$\rho$} & \textbf{MSE} & \textbf{$\rho$} & \textbf{MSE}\\
\midrule

&  & 50 & 0.36 & 0.66 & 0.34 & 1.05\\
%\cmidrule{3-7}
& & 100 & 0.37 & 0.66 & 0.36 & 1.04\\
%\cmidrule{3-7}
& CBOW & 300 & 0.38 & 0.65 & 0.37 & 1.03\\
%\cmidrule{3-7}
\multirow{2}{*}{FastText} & & 600 & 0.33 & 0.68 & 0.38 & 1.02\\
& & 1,000 & 0.39 & 0.64 & 0.41 & 0.99\\
\cmidrule{2-7}
& & 50 & 0.45 & 0.61 & 0.43 & 0.98\\
%\cmidrule{3-7}
& & 100 & 0.49 & 0.58 & 0.47 & 0.94\\
%\cmidrule{3-7}
& Skip-Gram & 300 & 0.55 & 0.53 & 0.40 & 1.02\\
%\cmidrule{3-7}
& & 600 & 0.40 & 0.64 & 0.40 & 1.01\\
& & 1,000 & 0.52 & 0.56 & 0.54 & 0.86\\
\midrule
 &  & 50 & 0.53 & 0.55 & 0.51 & 0.89\\
%\cmidrule{3-7}
& & 100 & 0.56 & 0.52 & 0.54 & 0.85\\
%\cmidrule{3-7}
& CBOW & 300  & 0.53 & 0.55 & 0.51 & 0.89\\
& & 600 & 0.49 & 0.58 & 0.53 & 0.87\\
\multirow{2}{*}{Wang2Vec}& & 1,000 & 0.50 & 0.57 & 0.53 & 0.87\\
\cmidrule{2-7}
& & 50 & 0.51 & 0.56 & 0.47 & 0.92\\
%\cmidrule{3-7}
&  & 100 & 0.54 & 0.54 & 0.50 & 0.89\\
%\cmidrule{3-7}
& Skip-Gram & 300 & 0.58 & 0.50 & 0.53 & 0.85\\
& & 600 & 0.59 & \textbf{0.49} & 0.54 & \textbf{0.83}\\
& & 1,000 & \textbf{0.60} & \textbf{0.49} & 0.54 & 0.85\\



\bottomrule
\end{tabular}}
\quad
\scalebox{0.73}{
\begin{tabular}[t]{llrrr|rr}
\toprule
\multicolumn{2}{c}{\multirow{2}{*}{\textbf{Embedding Models}}} & \multirow{2}{*}{\textbf{Size}} & \multicolumn{2}{c}{\textbf{PT-BR}} & \multicolumn{2}{c}{\textbf{PT-EU}} \\
\cmidrule{4-7}
 \multicolumn{2}{c}{} & & \textbf{$\rho$} & \textbf{MSE} & \textbf{$\rho$} & \textbf{MSE}\\
\midrule
& & 50 & 0.42 & 0.62 & 0.38 & 1.01\\
%\cmidrule{3-7}
& & 100 & 0.45 & 0.60 & 0.42 & 0.98\\
%\cmidrule{3-7}
GloVe & & 300 & 0.49 & 0.58 & 0.45 & 0.95\\
%\cmidrule{3-7}
%& & 400 & 0.50 & 0.57 & 0.46 & 0.95\\
& & 600 & 0.50 & 0.57 & 0.45 & 0.94\\
& & 1,000 & 0.51 & 0.56 & 0.46 & 0.94\\
\midrule
&  & 50 & 0.47 & 0.59 & 0.46 & 0.95\\
%\cmidrule{3-7}
&  & 100 & 0.50 & 0.57 & 0.49 & 0.91\\
%\cmidrule{3-7}
& CBOW & 300 & 0.55 & 0.53 & 0.54 & 0.87\\
%\cmidrule{3-7}
& & 600 & 0.57 & 0.51 & \textbf{0.55} & 0.86\\
%\cmidrule{3-7}
\multirow{2}{*}{Word2Vec} & & 1,000 & 0.58 & 0.50 & \textbf{0.55} & 0.86\\
\cmidrule{2-7}
& & 50 & 0.46 & 0.60 & 0.43 & 0.97\\
%\cmidrule{3-7}
& & 100 & 0.48 & 0.58 & 0.45 & 0.95\\
%\cmidrule{3-7}
& Skip-Gram & 300 & 0.52 & 0.56 & 0.48 & 0.93\\
%\cmidrule{3-7}
%& & 600 \cite{hartmann2016} & 0.58 & 0.50 & \textbf{0.55} & 0.83\\
%\cmidrule{3-7}
& & 600 & 0.53 & 0.54 & 0.50 & 0.92\\
%\cmidrule{3-7}
& & 1,000 & 0.54 & 0.54 & 0.50 & 0.91\\
\bottomrule
\end{tabular}}
\caption{Extrinsic evaluation on Semantic Similarity task.}
\label{tab:semantic_similarity_evaluation}
\end{table}
\section{Related Work}
  
The research on evaluating unsupervised word embeddings can be divided into intrinsic and extrinsic evaluations. The former relying mostly on word analogies (e.g. \cite{mikolovetal2013}) and measuring the semantic similarity between words (e.g. the WS-353 dataset \cite{finkelstein2001}), while extrinsic evaluations focus on practical NLP tasks  (e.g. \cite{2016nayak-veceval}). POS tagging, parsing, semantic similarity between sentences, and sentiment analysis are some commonly used tasks for this end.

%\todo[inline]{(erick) não entendi o que isso quer dizer... e acho que podemos pular esse trecho de qualquer forma: Intrinsic evaluations measure the quality of word embeddings usually through datasets of query terms; however, they have limitations since the datasets are precompiled and mostly limited to local metrics, such as relatedness \cite{conf/emnlp/SchnabelLMJ15}. ERICK e todos: vale a pena ler o artigo EVATUATION METHODS FOR UNSUPERVISED WORD EMBEDDINGS. Lá eles explicam os 4 tipos de avaliação intrínseca da relatedness: relatedness (correlação de com escores de humanos), analogy (popularizado pelo Mikolov), categorization (clusterização de palavras de categorias diferentes), selecional preference (quão tipico um substantivo está para um verbo ou para sujeito ou para um objeto). DAI o artigo apresenta os vários datasets típicos para o inglês: WordSim 353, MEN. E faz críticas sobre como são construídos esses datasets. Na verdade, nada foi comentado no nosso artigo sobre o dataset que [Rodrigues, 2016] usou. O que acham de explicar ???w}

%On the other hand, extrinsic evaluations seem to be the better option for novel word embeddings methods, although the problem here is to find a representative set of NLP tasks to include in a benchmark suite. \cite{2016nayak-veceval}, for example, propose tasks to test syntactic and semantic properties of the word embeddings, including POS tagging, chunking, named entity recognition, sentiment classification, question classification and phrase-level natural language inference in their suite of relevant downstream tasks.

%\todo[inline]{Aqui já tem que dar uma justificação do porque escolhemos as 2 tarefas}

%\todo[inline]{(Erick) Uma coisa que ficou faltando mencionar é que um tipo comum de avaliação em inglês são os testes de similaridade entre duas palavras, mencionados pelo Faruqui 2016.}

% To the best of our knowledge, only a few works attempted to evaluate Portuguese word embeddings.
% \cite{rodriguesetal2016} collected a corpus of Portuguese texts to train word embedding models using the Skip-Gram Word2Vec technique. Their corpus contains 1,723,693,241 tokens following the Universal Dependencies tokenization pattern. The authors also translated the benchmark of syntactic and semantic analogies developed by \cite{mikolovetal2013} and made it available\footnote{\url{https://github.com/nlx-group/lx-dsemvectors}} for both Brazilian and European Portuguese. They report a 52.8\% evaluation accuracy of their word embedding model in both syntactic and semantic analogies.

To the best of our knowledge, only a few works attempted to evaluate Portuguese word embeddings.
\cite{rodriguesetal2016} collected a corpus of Portuguese texts to train word embedding models using the Skip-Gram Word2Vec technique. The authors also translated the benchmark of word analogies developed by \cite{mikolovetal2013} and made it available\footnote{\url{https://github.com/nlx-group/lx-dsemvectors}} for both Brazilian and European Portuguese. The benchmark contains five types of semantic analogy: (i) common capitals and countries, (ii) all capitals and countries, (iii) currency and countries, (iv) cities and states, and (v) family relations. Moreover, nine types of syntactic analogy are also represented: adjectives and adverbs, opposite adjectives, base adjectives and comparatives, base adjectives and superlatives, verb infinitives and present participles, countries and nationalities (adjectives), verb infinitives and past tense forms, nouns in plural and singular, and verbs in plural and singular. They report a 52.8\% evaluation accuracy of their word embedding model in both syntactic and semantic analogies.

%\todo[inline]{Aqui já tem que mostrar os resultados comparativos com o nosso córpus}

\cite{sousa2016} investigated whether Word2Vec (CBOW and Skip-Gram) or GloVe performed best on the benchmark in \cite{rodriguesetal2016}. The author compiled a sample of texts from Wikipedia in Portuguese, searching for articles related to teaching, education, academics, and institutions. The best results were obtained using Word2Vec CBOW to train vectors of 300 dimensions. This model achieved an accuracy of 21.7\% on syntactic analogies, 17.2\% on semantic analogies and 20.4\% overall.

\cite{Fonseca2015} compared the performance of three different vector space models used for POS tagging with a neural tagger. They used Word2Vec Skip-Gram, HAL, and the neural method from \cite{collobertetal2011}; Skip-Gram obtained the best results in all tests.

Concerning the differences between embeddings obtained from Brazilian and European Portuguese texts, \cite{Fonseca2016} present an extrinsic analysis on POS tagging. They trained different embedding models; one with only Brazilian texts, one with only European ones and another with mixed variants; and trained neural POS taggers which were evaluated on Brazilian and European datasets. One of their findings is that, as a rule of thumb, the bigger the corpus in which embeddings are obtained, the better. Additionally, mixing both variants in the embedding generation did not decrease tagger performance in any of the POS test sets. This supports the hypothesis that a single, large corpus comprising Brazilian and European texts can be useful for most NLP applications in Portuguese.
\section{Conclusions and Future Work}

%In this paper, we collected a large corpus of Portuguese texts in order to train word embedding models using four different techniques. 
In this paper, we presented the word embeddings we trained using four different techniques and their evaluation.
All trained models are available for download, as well as the script used for corpus preprocessing. The results obtained from intrinsic and extrinsic evaluations were not aligned with each other, contrary to the expected. GloVe produced the best results for syntactic and semantic analogies, and the worst, together with FastText, for both POS tagging and sentence similarity. These results are aligned with those from \cite{repeval:16}, which suggest that word analogies are not appropriate for evaluating word embeddings. Overall, Wang2Vec vectors yielded very good performance across our evaluations, suggesting they can be useful for a variety of NLP tasks.
% A sentença abaixo ocupava 6 linhas e estava cheia de achismos.... sou contra incluí-la no texto final
%As future work, we intend to try different tokenization and normalization patterns, e.g., the one used in the project Universal Dependencies, to evaluate how much tokenization affects the word embedding representativeness in different tasks or better normalization strategies for numbers which should be verbalized non-numerically as well as internet language since these patterns can change semantic and syntactic relationships when properly identified. 
As future work, we intend to try different tokenization and normalization patterns, and also to lemmatize certain word categories like verbs, since this could significantly reduce vocabulary, allowing for more efficient processing. An evaluation with more NLP tasks would also be beneficial to our understanding of different model performances.

%\todo[inline]{(erick) até concordo que tokenização pode dar algumas diferenças interessantes, mas a ponto de ser mencionado nas conclusões em lugar de outras coisas mais interessantes?? (SANDRA) Erick: elenque os trabalhos fututos interessantes! O artigo é seu também :). Chris: eu acho que alem de tokenizacao, lemmatizacao e normalizacao sao muito importantes, os verbos se exandem muito no vocabulario e a gente perde toda semantica onde tem datas, horarios, medidas, abreviacoes, etc. Nathan: lematizar vai perder a riqueza dos verbos mas normalizar datas, horários, dinheiro podem melhorar a semântica das embeddings. Hoje apenas normalizamos números.}

%Interestingly, if Pinocchio said ``my nose will grow now'', this would cause a paradox. lol fun fan fun


\section*{Acknowledgements}

This work was supported by CNPq, CPqD and FAPESP (PIPE-PAPESP project nº 2016/00500-1).

\bibliographystyle{sbc}
\bibliography{references}

\end{document}
